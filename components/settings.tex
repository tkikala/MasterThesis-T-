\usepackage[utf8]{inputenc}
\usepackage{a4}
\usepackage{titlesec}
\usepackage[none]{hyphenat} %hyphenation
\sloppy
\usepackage{parskip} %no indentation after paragraphs
% \usepackage{umlaute}
\usepackage{afterpage} %for using \afterpage{\clearpage} (don't push images to the end of a chapter)
\usepackage{makeidx}
\usepackage[numbers]{natbib}
\usepackage{graphicx}
\usepackage{style/picins} %provides precise control over the placement of inline graphics
\usepackage{setspace}
\usepackage{titlesec}
\usepackage{dsfont} %math symbols
\usepackage{tabularx}
\usepackage{floatflt} %float text around figures and tables
% Florian Schulze, 06.06.2012
% v1.0, latest edit: 06.06.2012

\usepackage{enumitem} %resume counting from previous enumerate block
\usepackage{amsmath,amssymb}
\usepackage[format=default,font=footnotesize,labelfont=bf]{caption}
\usepackage{listings} %for listing source code
\usepackage{color}
\usepackage{algpseudocode} %for listing pseudocode
\usepackage{algorithm} %wrap algpseudocode and enrich with label etc.
\usepackage{float} % for [H] after floats
\usepackage{url}
\usepackage{hyperref}
\usepackage[toc,page]{appendix}
\usepackage{array}
\usepackage{multirow}
\usepackage{longtable}
\usepackage{subcaption}
% \usepackage{dingbat}
% \usepackage{longtable}
% \usepackage[acronym]{glossaries}
% \usepackage[
% nonumberlist, %do not show page numbers
% acronym,      %generate acronym listing   -> Not used in this example (see line with %%% )
% toc,          %show listings as entries in table of contents
% section]      %use section level for toc entries
% {glossaries}
\usepackage{geometry}
 \geometry{
 a4paper,
 left=45mm,
 top=45mm,
 bottom=45mm,
 }

\usepackage[acronym, toc]{glossaries}

\titleformat{\chapter}{\normalfont\huge\bfseries}{\thechapter.}{20pt}{\huge}

\titleformat{\paragraph}[hang]{\normalfont\bfseries}{\theparagraph}{.5em}{}

\lstset{
    basicstyle=\ttfamily\footnotesize,
    language=Python,
    breaklines=true,
    frame=single,
    breakatwhitespace=false, 
    % numbers=left,
    literate=::1
}
\DeclareCaptionLabelFormat{andtable}{#1~#2  \&  \tablename~\thetable}