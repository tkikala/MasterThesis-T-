\newpage
\addcontentsline{toc}{chapter}{Abstract}
\thispagestyle{empty}
\hoffset=0mm
% The abstract goes here
\begin{center}
    {\Large \bf Abstract}
\end{center}
\begin{spacing}{1.35}

\par
The mobile sector has developed at an extremely fast pace in the latest decades. An increasing amount of new devices, as well as new frameworks and programming languages for developing applications for mobile platforms are being introduced every year. As testing is one of the most important parts in a products life-cycle, there is a need to improve its speed, effectiveness and cost-efficiency. The automation of test execution has been proven to be a very effective approach for reducing testing time, but there is still enough opportunities to improve current testing processes. These include static test suites, reducing the time required to achieve complete test execution automation amongst others. \acrlong{mbt} (\acrshort{mbt}) raises test automation to a new level while providing the opportunity to not only automate the test execution, but also the test generation process. It addresses above mentioned issues by generating unique tests for each test run using the behavioral models of \acrlong{aut} (\acrshort{aut}). This is a valid reason for Brainloop to try out \acrshort{mbt} for one of its mobile applications, the \acrlong{bsc} (\acrshort{bsc}), and study the improvements in software quality, cost-effectiveness of this approach as well as the vision for effective integration of \acrshort{mbt} into its agile environment.

\par
In the first phase of this thesis we undertake a literature review to identify the most suitable tool for applying \acrshort{mbt} for \acrshort{bsc} according to requirements from Brainloop. As a result, we choose the open-source tool Graphwalker which is being developed and used at Spotify.

\par
The next phase is dedicated to modeling the behaviour of \acrshort{bsc} in an abstracted manner. For this purpose, yEd Graph Editor is used to create models supported by Graphwalker. Overall, 33 models are created to describe significant part of the functionality of \acrshort{bsc}.

\par
In the last phase we generate test sequences with different coverage criteria and execute tests manually on one physical device. As a result, some issues not detected during the current testing process were discovered. We also provide the vision of gradually integrating \acrshort{mbt} into the agile scrum process which is currently used by Brainloop.

\end{spacing}
