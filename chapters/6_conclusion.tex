\chapter{GENERAL DISCUSSION}
\label{chapter:conclusion}

\section{Vision for integrating \acrshort{mbt} into the scrum}

\par
A couple of different approaches could be applied for integrating \acrshort{mbt} into the current scrum process in Brainloop. One of them is integrating it in a very rigid way, directly discarding current testing process, training quality assurance staff to get familiar with modeling concepts, apply the basis provided by this thesis and start improving models and testing \acrshort{bsc} based on them. This would be very harmful and time consuming, as it will require too many radical changes at the same time. Another suggestion which is provided by CA Technologies, is that product owners need to deliver features already in modeled representation and groom it in a refinement meeting together with the team. That also would require training of the complete team together with product owners to adapt to new approach.

\par
Our suggestion is to gradually apply \acrshort{mbt} in current testing process. The process should start by giving short training to the testers on \acrshort{mbt}. One feature should be chosen and team needs to start enhancing existing models for it with their existing knowledge about that feature. As a next step, an adapter should be written for the test execution engine, Xamarin.UITest framework, currently used in \acrshort{bsc} development team to connect tests generated by Graphwalker with the respective automated module. Next step would be gradual automation of execution for states and transitions within base and navigation layers as well as in the chosen feature. All of this should happen along with the current testing process, so that testing team gradually adapts to the new approach and that this does not harm the quality of the product during the adaption time. Once proven that the process is working and the team is comfortable with it, instead of documenting static test cases in issue tracker software, the team will need to create model of the feature and adapt other models according to the user story provided by product owner. If the management insists on some kind of textual description of abstract actions and states, team can also document this information separately in issue tracker system, but this will still be much better than static test cases as if update is required, it is updated in one place and then used in all test cases. This will result in smooth and eventually complete transition from current testing process to \acrshort{mbt}.

\section{Limitations and future work}

\par
We would like to point out two main topics where we bumped into the limitations in this thesis. Firstly, Graphwalker's coverage criteria was mostly limited to random. It would have been very helpful if it had supported New York Street Sweeper Algorithm generating shortest path through directed graph after traversal. There is high chance that if we had had this algorithm up and running, we would have been able to execute tests with complete coverage of all 33 models together, as the resulting sequence would reduce tremendously. Another limitation is related to the off-line \acrshort{mbt} approach itself. While creating models and executing tests, we realized that a feedback loop between the test execution and test generation would be very useful feature, as it does not limit us with already predefined variables and settings and can make our tests much more realistic and not bound to pre-configurations.

\par
As a future work, of course, the first thing in consideration would be to integrate \acrshort{mbt} into the testing process in Brainloop. This would itself bring up new processes, timelines and experiences.  Secondly, implementing the New York Street Sweeper Algorithm for \acrshort{efsm}s in Graphwalker would be huge contribution and the biggest token of gratitude we can extend to its open-source community. Last, but not least, after the complete set of transitions and state verifications is automated, the first consideration should be moving from off-line to on-line \acrshort{mbt}. If Graphwalker is chosen as a tool for \acrshort{mbt} in Brainloop, transition to on-line \acrshort{mbt} would require team to adopt usage of Java or Python programming language, but this can be considered as late future.

\section{Conclusion}

\par
We aimed at applying \acrshort{mbt} to the iOS application \acrshort{bsc} created and maintained at Brainloop, providing results from testing as well as a vision for integrating \acrshort{mbt} into the scrum process in Brainloop.

\par
Current \acrshort{mbt} approaches and tools were examined through a literature review, during which multiple papers, case-studies and theses were analyzed. This resulted in 3 possible paradigms for modeling and 6 possible tools applying these paradigms. Based on the requirements provided by Brainloop, that all the quality assurance staff needs to be able to easily apply the new approach, we chose the tool Graphwalker to proceed with creation of behavioral models of \acrshort{bsc}.

\par
We modeled significant part of the \acrshort{bsc} with the predefined level of abstraction using yEd Graph Editor. This resulted in 33 different models with overall 312 states and 566 transitions. We structured these models in layers, where the top layer contains models for authentication, next layer contains models for navigation through different tabs and last layer contains all other operational models structured under each tab.

\par
The next step was the generation of tests from the designed models using Graphwalker. To ease this process, we created a small \acrshort{gui} to pass the test generation criteria to Graphwalker and to display a easy to read response. As the complete coverage of all models together was not feasible due to the enormously long generated test sequence, we came up with notion of \acrshort{mrp} and generated tests using it for every model in the third layer of our layered model structure.

\par
We executed tests generated on one device taking approximately 39 hours, during which 4196 states and transitions were respectively verified and executed and 38 issues were detected. After looking these up in the issue tracker, we found that 30 issues were new while 8 were already known. The quality assurance team took cognizance of it.

\par
Finally, the vision for integrating \acrshort{mbt} into the current scrum process gradually and without requiring radical changes to the current testing process was presented to Brainloop.
 








 


