\chapter{TEST EXECUTION AND RESULTS}
\label{chapter:testing_results}

Tests were executed only on one device, iPhone 6s owned by Brainloop. Test execution took roughly 39 hours and detected 38 issues in total. Unfortunately, we cannot describe them or mention their severity in the thesis as Brainloop wanted it to remain confidential. A complete test report will be handed over to Brainloop as one artifact of this thesis. We try to provide some more insight on the execution of the tests in following section.

\section{Results and Spent Effort for each model tested with MRP}
Table \ref{tab:Test_Results} points out how a total of 4196 items (either states or transitions) were distributed per model, the time spent on executing them and the number of issues that were detected by tests generated from each model.

\begin{table}[]
    \centering
    \begin{tabular}{|l|p{2cm}|p{2cm}|p{2cm}|}
        \hline
        \textbf{Models Tested with \acrshort{mrp}} & \textbf{Parsed Test Sequence Length in Items} & \textbf{Time Spent on Execution in Hours} & \textbf{Number of Detected Issues}\\
        \hline
        \textit{Base and Navigation layers} & 822 & 3 & 2 \\
        \hline
        Datarooms & 152 & 2 & 3 \\
        \hline
        Dataroom Context Menu & 83 & 1 & 1 \\
        \hline
        Documents & 67 & 0.25 & 0 \\
        \hline
        Recently Changed & 67 & 1.5 & 4 \\
        \hline
        Recently Viewed & 95 & 1 & 1 \\
        \hline
        Events & 215 & 2 & 2 \\
        \hline
        All Version View & 28 & 0.5 & 2 \\
        \hline
        File Context Menu - Downloaded & 86 & 1 & 0 \\
        \hline
        File Context Menu - Not Downloaded & 101 & 1 & 0 \\
        \hline
        File Context Menu - Downloaded PDF & 87 & 1 & 0 \\
        \hline
        Share Review & 157 & 2 & 3 \\
        \hline
        Send Securely & 218 & 2 & 2 \\
        \hline
        Original File Viewer & 37 & 0.5 & 0 \\
        \hline
        PDF Viewer Operations & 792 & 4 & 7 \\
        \hline
        PDF Viewer Annotations & 252 & 3 & 3 \\
        \hline
        Settings & 58 & 0.75 & 0 \\
        \hline
        Access Code Settings & 30 & 0.5 & 0 \\
        \hline
        Application View Settings & 67 & 0.75 & 0 \\
        \hline
        Delete Settings & 32 & 0.5 & 0 \\
        \hline
        Download Settings & 50 & 1 & 1 \\
        \hline
        Security Settings & 132 & 2.5 & 0 \\
        \hline
        Change Account Password & 62 & 1.5 & 1 \\
        \hline
        Servers and Datarooms Settings & 107 & 2.5 & 0 \\
        \hline
        Add Dataroom Server & 30 & 0.75 & 0 \\
        \hline
        Remove Dataroom Server & 72 & 0.75 & 0 \\
        \hline
        Support & 64 & 0.5 & 3 \\
        \hline
        Votes & 230 & 2 & 1 \\
        \hline
        \textbf{Total} & \textbf{4196} & \textbf{39} & \textbf{38}\\
        \hline
    \end{tabular}
    \caption{Test Execution Results }
    \label{tab:Test_Results}
\end{table}

\section{Comparison to results from current testing process}

\par
After comparing our list of issues with already reported bugs in the internal issue tracker, we found out that 30 of 38 bugs were not reported there, while 8 of them were already reported. This may be attributed to the deep exploration of the modeled functionality by Graphwalker.

\par
However, we did not detect many bugs which were already reported during the current testing process. This may be due to couple of reasons. First of all, a part of the functionality was omitted while modeling, as mentioned in Chapter \ref{chapter:modeling}. Another reason is that \acrshort{bsc} is strongly affected by changes in configuration of \acrshort{bdrs}. We performed our tests with only one configuration of \acrshort{bdrs} and one type of dataroom. Finally, the generated tests against \acrshort{bsc} were executed only on one device while the application might have behaved differently on other devices.

\par
To sum up, these were the results obtained from the first drafts of our models. Once Brainloop's quality assurance team is brought on-board with this project and the modeling procedure is explained to them, the models would be updated and extended to test far more functionalities of the application. There are still many parameters and corner cases/configurations of \acrshort{bdrs} which only experienced members of the \acrshort{bsc} team would be capable of modeling. I would refer again to Apfenbaum's comment \cite{Apfenbaum_MBT}: this information is gold!

