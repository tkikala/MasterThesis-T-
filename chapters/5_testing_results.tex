\chapter{TEST EXECUTION AND RESULTS}
\label{chapter:testing_results}

Tests were executed only on one device, iPhone 6s owned by Brainloop. Test execution took roughly 39 hours and detected 38 issues overall. Unfortunately, we cannot mention description and severity of the issues in thesis, as it is confidential information of Brainloop. Complete test report will be handed over to Brainloop as one artifact of this thesis. More detailed information about execution can be observed in following section.


\section{Results and Spent Effort for each model tested with MRP}
Table \ref{tab:Test_Results} points out how total 4196 items were distributed per model, how much time was spent on executing and how many issues were detected by tests generated from each model.

\begin{table}[]
    \centering
    \begin{tabular}{|l|p{2cm}|p{2cm}|p{2cm}|}
        \hline
        \textbf{Models Tested with \acrshort{mrp}} & \textbf{Parsed Test Sequence Length in Items} & \textbf{Time Spent on Execution in Hours} & \textbf{Number of Detected Issues}\\
        \hline
        \textit{Base and Navigation layers} & 822 & 3 & 2 \\
        \hline
        Datarooms & 152 & 2 & 3 \\
        \hline
        Dataroom Context Menu & 83 & 1 & 1 \\
        \hline
        Documents & 67 & 0.25 & 0 \\
        \hline
        Recently Changed & 67 & 1.5 & 4 \\
        \hline
        Recently Viewed & 95 & 1 & 1 \\
        \hline
        Events & 215 & 2 & 2 \\
        \hline
        All Version View & 28 & 0.5 & 2 \\
        \hline
        File Context Menu - Downloaded & 86 & 1 & 0 \\
        \hline
        File Context Menu - Not Downloaded & 101 & 1 & 0 \\
        \hline
        File Context Menu - Downloaded PDF & 87 & 1 & 0 \\
        \hline
        Share Review & 157 & 2 & 3 \\
        \hline
        Send Securely & 218 & 2 & 2 \\
        \hline
        Original File Viewer & 37 & 0.5 & 0 \\
        \hline
        PDF Viewer Operations & 792 & 4 & 7 \\
        \hline
        PDF Viewer Annotations & 252 & 3 & 3 \\
        \hline
        Settings & 58 & 0.75 & 0 \\
        \hline
        Access Code Settings & 30 & 0.5 & 0 \\
        \hline
        Application View Settings & 67 & 0.75 & 0 \\
        \hline
        Delete Settings & 32 & 0.5 & 0 \\
        \hline
        Download Settings & 50 & 1 & 1 \\
        \hline
        Security Settings & 132 & 2.5 & 0 \\
        \hline
        Change Account Password & 62 & 1.5 & 1 \\
        \hline
        Servers and Datarooms Settings & 107 & 2.5 & 0 \\
        \hline
        Add Dataroom Server & 30 & 0.75 & 0 \\
        \hline
        Remove Dataroom Server & 72 & 0.75 & 0 \\
        \hline
        Support & 64 & 0.5 & 3 \\
        \hline
        Votes & 230 & 2 & 1 \\
        \hline
        \textbf{Total} & \textbf{4196} & \textbf{39} & \textbf{38}\\
        \hline
    \end{tabular}
    \caption{Test Execution Results }
    \label{tab:Test_Results}
\end{table}

\section{Comparison to results from current testing process}

\par
After comparing our list of issues with already reported bugs in issue tracker software, we found out that 30 of 38 bugs were not reported there, while 8 of them were already reported. The reason behind it can be explained with the deep exploration of the modeled functionality using Graphwalker. Basically, whatever is modeled gets tested very extensively.

\par
On the other hand, we did not detect many bugs which were already reported during the current testing process. This also can be explained with couple of reasons. First of all, part of the functionality was omitted while modeling, mentioned in 3rd section. Another reasons is that, models created during the thesis might not cover the functionality completely, as \acrshort{bdrs} provides thousands of different combinations of configurations and application might react to it differently. Last, but not least, generated tests against \acrshort{bsc} were executed only on one device while on different devices application might behave differently.

\par
To sum up, these are the result of first drafts of our models. After describing the concept of modeling to the quality assurance team already involved in testing of \acrshort{bsc}, with their contribution and brainstorming, models will be extended to test far more combinations of application and will get enhanced with more parameters, as they are people who know all hidden behind configuration of \acrshort{bdrs} which affect \acrshort{bsc}. I would refer again to Apfenbaum's comment \cite{Apfenbaum_MBT}, this information is gold!
